%
\documentclass[12pt]{article}

% The usual packages
\usepackage{fullpage}
\usepackage{breakcites}
\usepackage{setspace}
\usepackage{endnotes}
\usepackage{float}
\usepackage{amsmath}
\usepackage{amsfonts}
\usepackage{amssymb}
\usepackage{rotating}
\usepackage{dcolumn}
\usepackage{longtable}
\usepackage{microtype}
\usepackage{graphicx}
\usepackage{hyperref}
\usepackage[usenames,dvipsnames]{color}
\usepackage{url}
\usepackage{natbib}
\usepackage{framed}
\usepackage{lipsum}
\usepackage[font=small,labelfont=sc]{caption}
\restylefloat{table}
\bibpunct{(}{)}{;}{a}{}{,}

% Set paragraph spacing the way I like
\parskip=0pt
\parindent=20pt

% Define mathematical results
\newtheorem{lemma}{Lemma}
\newtheorem{proposition}{Proposition}
\newtheorem{theorem}{Theorem}
\newtheorem{claim}{Claim}
\newenvironment{proof}[1][Proof]{\begin{trivlist}
\item[\hskip \labelsep {\bfseries #1}]}{\end{trivlist}}
\newenvironment{definition}[1][Definition]{\begin{trivlist}
\item[\hskip \labelsep {\bfseries #1}]}{\end{trivlist}}
\newenvironment{example}[1][Example]{\begin{trivlist}
\item[\hskip \labelsep {\bfseries #1}]}{\end{trivlist}}
\newenvironment{remark}[1][Remark]{\begin{trivlist}
\item[\hskip \labelsep {\bfseries #1}]}{\end{trivlist}}

% Set up fonts the way I like
\usepackage{tgpagella}
\usepackage[T1]{fontenc}
\usepackage[bitstream-charter]{mathdesign}



%% Set up lists the way I like
%Redefine the first level
\renewcommand{\theenumi}{\arabic{enumi}.}
\renewcommand{\labelenumi}{\theenumi}
%Redefine the second level
\renewcommand{\theenumii}{\alph{enumii}.}
\renewcommand{\labelenumii}{\theenumii}
%Redefine the third level
\renewcommand{\theenumiii}{\roman{enumiii}.}
\renewcommand{\labelenumiii}{\theenumiii}
%Redefine the fourth level
\renewcommand{\theenumiv}{\Alph{enumiv}.}
\renewcommand{\labelenumiv}{\theenumiv}

% Create footnote command so that my name
% has an asterisk rather than a one.
\long\def\symbolfootnote[#1]#2{\begingroup%
\def\thefootnote{\fnsymbol{footnote}}\footnote[#1]{#2}\endgroup}

\hypersetup{
 pdftitle={Strategic Mobilization}, % title
 pdfauthor={Carlisle Rainey}, % author
 pdfkeywords={mobilize} {mobilization} {turnout} {proportional} {singe-member district} {PR} {SMD} {SMDP}
 pdfnewwindow=true, % links in new window
 colorlinks=true, % false: boxed links; true: colored links
 linkcolor=BrickRed, % color of internal links
 citecolor=BrickRed, % color of links to bibliography
 filecolor=BrickRed, % color of file links
 urlcolor=BrickRed % color of external links
}

\begin{document}

\begin{center}
{\LARGE Strategic Mobilization}\\\vspace{2mm}
{\large Why Proportional Representation Decreases Voter Mobilization}\\
\vspace{3mm}
Carlisle Rainey\symbolfootnote[2]{Carlisle Rainey is Assistant Professor of Political Science, University at Buffalo, SUNY, 520 Park Hall, Buffalo, NY 14260 (\href{mailto:rcrainey@buffalo.edu}{rcrainey@buffalo.edu}). I thank John Ahlquist, William Berry, Scott Clifford, Matt Golder, Sona Golder, Jens Grosser, Bob Jackson, John Barry Ryan, and Dave Siegel for their comments on previous drafts. The analyses presented here were conducted with \texttt{R} 3.1.0 and \texttt{JAGS} 3.3.0. All data and computer code necessary for replication are available at \href{https://github.com/carlislerainey/strategic-mobilization}{github.com/carlislerainey/strategic-mobilization}.}\\\singlespace
\end{center}

% Remove page number from first page
%\thispagestyle{empty}

% Abstract
{\centerline{\textbf{Abstract}}}
\begin{quote}\noindent Many scholars suggest that proportional representation increases party mobilization by creating nationally competitive districts that give parties an incentive to mobilize everywhere. This paper provides theoretical and empirical arguments that bring this claim into question. I propose, unlike earlier scholars, that the positive effect of district competitiveness on party mobilization efforts increases as electoral districts become more \textit{dis}proportional, arguing that disproportionality itself encourages mobilization by exaggerating the impact of competitiveness on mobilization. Individual-level survey data from national legislative elections show that competitiveness has a much larger positive effect on parties' mobilization efforts in single-member districts than in proportional districts. Contrary to prior literature, these results suggest proportional electoral rules give parties no strong incentive to mobilize anywhere.\end{quote}
\thispagestyle{empty}

%\end{document} % uncomment to make a title-page with author info

\newpage
\doublespace
% Set first page of text as page 1. I don't care for this
%   feature because then the page numbers don't correspond
%   to the pdf pages.
%\setcounter{page}{1}

\section*{Introduction}

Does proportional representation cause parties to mobilize more voters? Many studies of electoral systems suggest proportional electoral rules do lead to greater mobilization (and thus increased turnout). However, more recent work argues that the evidence is too limited and the theories too under-developed to support this conclusion. In particular, \cite{BlaisAarts2006} suggest that political scientists cannot have confidence that proportional rules cause higher turnout until scholars better understand the mechanism linking the two. 

Several explanations have emerged that attempt to explain the observation of higher turnout under proportional representation (PR) rules (for an overview, see \citealt{BlaisAarts2006}). The most theoretically compelling focuses on the frequent emergence of non-competitive electoral districts in single-member district plurality (SMDP) systems. This explanation suggests that parties (or candidates and activists more broadly) exert greater mobilization efforts under PR rules than under SMDP rules because PR rules, on average, create more competitive districts \citep{Cox1999}. Many scholars take as given that PR rules create more competitive districts, although recent work brings this common assumption into question \citep{BlaisLago2009}. More competitive districts, in turn, provide parties a strong incentive to mobilize voters.

A large literature confirms that turnout (e.g. \citealt{RosenstoneHansen1993}) and mobilization \citep{CoxMunger1989, KarpBanducciBowler2007} are higher in more competitive districts, but this relationship has only been examined in SMDP systems. Research examines the relationship between competition and turnout in Canada \citep{MatsusakaPalda1993, Endersbyetal2002} and Britain \citep{DenverHands1974, DenverHands1985}, consistently finding higher turnout in more competitive districts. Further, \cite{KarpBanducciBowler2007} estimate the effect of competitiveness in the United States, Britain, New Zealand, Canada, and Australia, and find a substantial effect of competitiveness in each country. Their empirical analysis and conceptual approach do not allow competitiveness to vary across districts within PR systems. Recent work on the conceptualization and measurement of competitiveness shows that it can and does vary across districts within PR systems \citep{GrofmanSelb2009, Selb2009, BlaisLago2009}.

Despite the emphasis that previous work places on increased competition under PR rules, I argue that competitiveness, while it might be higher under PR rules (though see \citealt{BlaisLago2009}), should not have a substantively important effect in these systems. This argument is closely related to the previous. Districts that use winner-take-all rules disproportionately reward the winner. From the perspective of the trailing candidate, disproportional rewards become increasingly obtainable as the race narrows. For the leader, the rewards become increasingly in doubt. In this situation, the incentive to mobilize a few extra voters is large. Because there are no disproportionate rewards in proportional districts to encourage fierce competition over voters, we should not expect competitiveness to play as important a role in proportional systems. Rather than PR rules creating an incentive for parties to mobilize everywhere, PR rules create no strong incentive to mobilize anywhere.

The resolution to the debate over whether PR rules cause more mobilization efforts and higher turnout has important implications for representative democracy. As noted by many previous studies \citep{WolfingerRosenstone1980, RosenstoneHansen1993, BradyVerbaSchlozman1995}, wealthier, more educated, and higher SES citizens turn out at a greater rate than other citizens. Because elected officials have an incentive to respond to voters rather than the citizens as a whole \citep{Downs1957}, the resulting policies reflect the interests of only some citizens. While scholars disagree over the severity of this problem \citep{BerelsonLazarsfeldMcPhee1954, Lijphart1997, Teixeira1992}, most agree that low turnout poses an obstacle to an ideal democracy. Indeed, Arend Lijphart (1997) calls unequal participation ``democracy's unresolved dilemma,'' and suggests PR electoral institutions as a resolution.

Further, \cite{Sniderman2000} points out that parties play an important role in structuring the political world, allowing relatively uninformed voters to make sense of it. Political scientists know a great deal about how many parties are likely to emerge in a political system (e.g. \citealt{Cox1997, ChhibberKollman1998, ClarkGolder2006}) and where these parties are likely to position in the ideological space (e.g. \citealt{Cox1990, KollmanMillerPage1992, AdamsMerrillGrofman2005}), but political scientists know relatively less about what rules give parties an incentive to mobilize voters, making political participation less costly and providing voters with the information necessary to make good choices \citep{Downs1957}.

I make three contributions in this paper. First, I argue that previous models of electoral competition miss two important and related points: (1) Disproportionality itself provides parties a strong incentive to mobilize and (2) disproportionality also exacerbates the effect of competitiveness on mobilization efforts. Together, these points suggest a reevaluation of the claim that proportionality encourages parties to mobilize voters because it creates ``nationally competitive districts''. Second, unlike most previous work, I recognize that competitiveness can vary in multimember districts and use a recently developed measure of district competitiveness to directly compare the effect of competitiveness on mobilization efforts in PR and SMDP systems. Finally, I test comparative statics with a Bayesian hierarchical model. The empirical results confirm the theoretical claim that disproportionality itself gives parties a strong incentive to mobilize and increases the impact of competitiveness on mobilization efforts.

The paper begins with a formal theoretical discussion that characterizes the incentives of parties to mobilize voters as district competitiveness and disproportionality vary, finding that disproportional rules increase the incentives to mobilize. I then proceed with empirical tests of the implications derived from the formal model, using a recently-introduced measure of district competitiveness and a Bayesian multilevel modeling strategy. Overall, I find that the observed data are consistent with the theoretical model. 

\section*{Parties' Incentives to Mobilize}

% This paragraph describes why it makes sense to shift our focus from the pivotal voter to the pivotal party.

Consistent with recent trends in the literature examining comparative electoral institutions \citep{Cox1999, Selb2009}, the theory presented below focuses on the elite response to the electoral environment. \cite{DenverHands1974} suggest that ``higher turnout in marginal seats is rarely the product of `rational' appreciation of the situation by \textit{voters}, but results from \textit{parties} creating greater awareness amongst voters or simply cajoling them into going to the polls [italics mine].''\footnote{Cited in \cite{Cox1999}.} Citizens respond to party mobilization, which occurs when mobilization might influence the outcome of the election. This shifts the focus from the utility calculations of voters to the utility calculations of elites \citep{CoxMunger1989, Cox1999}. Thus, races in which the outcome seems certain should receive little attention from either party. The apparent winner and loser have little incentive to devote scarce resources in a non-competitive district. However, in races with an uncertain outcome, candidates and parties have strong incentives to invest resources into mobilization, since their efforts might prove pivotal. Scholars originally developed this logic to explain variation across districts \textit{within} countries, primarily the United States, but \cite{Cox1999} generalizes this logic. Building on Cox and extending his argument, the formal model focuses on characteristics of the district that affect parties' incentives to mobilize.

\subsection*{The Theoretical Model}

To model electoral contests as disproportionality and competitiveness vary, I take insights from the theory of auctions (or contests) developed in economics \citep{Tullock1980, Hirshleifer1989} and apply them to electoral competition.\footnote{In particular, electoral competition is a form of all-pay auction, in which each bidder pays her bid, but only the highest bidder receives the prize. Similarly, political parties invest scare resources into political campaigns, but only the party that receives the most votes wins the seat. Here I consider a form of all-pay auctions in which the bidders compete over a divisible prize, and consider the equilibria as the cost of bidding (or equivalently, the value of the item) and the share disproportionality of the payoff vary.} The formal model of partisan competition that forms the key argument of this paper considers two parties,\footnote{In the online mathematical appendix, I use a more complicated but less general model to consider briefly the implications of expanding the competition to include $n$ parties, where $P = \{1, 2,..., n\}$. However, the two-party model is sufficient to develop the intuition for the reader. Also, the term ``party'' might be thought of broadly in the model, since individual candidates and activists face the same incentives.} $w$ and $s$,  competing over a single seat or set of seats with a value normalized to one.\footnote{At first, this assumption seems to devalue seats in a multimember district. For example, this assumes the value of one seat in a seven member district is the \begin{scriptsize}$\dfrac{1}{7}$\end{scriptsize} of the value of the single seat in a single member district. However, this assumption is not unwarranted. Imagine a single-member district. There are two ways to increase the number of seats. First, we might simply increase the number of seats in this district (an presumably other districts, to ensure equal representation), which subsequently increases the number of seats in the legislature. This is the process the model literally represents. However, a second interpretation is also possible. Rather than increase the number of seats in the legislature, we might instead decrease the total number of districts, increase the geographic area and number of citizens contained in each district, and keep the total number of legislators the same. In this case, the value of each seat says the same so that the total prize in seven-member district (seven seats) is seven times as valuable as a the total prize in a single member district (one seat). However, while the value of the total prize increases, so does the cost of mobilizing $100\epsilon$ percent of the citizens in that district. The increase in cost is approximately proportional to the increase in value of the seats. In the previous example, it is plausible to assume that it is seven times costlier to mobilize 30\% of the citizens in a district with seven times as large. Rather than build this assumption explicitly into the model, I have chosen to present the more parsimonious form. However, the conclusions are identical as long as costs of mobilization increase proportionally with the value of the seats in a district.\label{fn:value}}  Each party $i$  choose s toe exert a nonnegative amount effort, $\epsilon_i$, in order to win a share of the seats available in the district. The first party, $w$, is the weaker of the two parties and unable to mobilize voters as efficiently as $s$. This is captured by a parameter, $\beta \in [0, 1]$, such that equal efforts from the parties earns $w$ only $\beta$ times as much support as $s$ earns, in expectation, so that efforts of $\epsilon_w$ and $\epsilon_s$ earn the parties an average of $\beta\epsilon_w$ and $\epsilon_s$ support, respectively. Each party knows its own strength and the strength of the other party. The parameter $\beta$ is taken as an indicator of the competitiveness of the contest. If the two parties have the same ability to mobilize supporters, then $\beta = 1$ and the race is perfectly competitive. However, as $w$ becomes increasingly weak, $\beta$ moves toward zero and the race becomes less competitive.

Further, each contest varies in its disproportionality. This variation is captured by the parameter $\delta$. The expected share of the seats earned by $w$ is given by $\pi_w = \dfrac{(\beta \epsilon_w)^\delta}{(\beta \epsilon_w)^\delta + \epsilon_s^\delta}$, and the expected share of seats earned by $s$ is given by $\pi_s = \dfrac{\epsilon_s^\delta}{(\beta \epsilon_w)^\delta + \epsilon_s^\delta}$. As $\delta$ increases, the party that receives the greatest support receives an increasingly disproportionate share of the prize. Notice that $\pi_w$ and $\pi_s$ represent the expected vote share of the parties. The expectation averages over the parties subjective uncertainty about the outcome for a given level of mobilization effects. This assumption ``smooths'' the utility function, so that rather than a discontinuous bump in utility occurring when parties earn an additional seat, the increase is smooth.  

Finally, each party pays a cost equal to the size of the mobilization effort.\footnote{I assume a fixed cost that might plausibly be thought to vary with district magnitude. For example, it makes sense that mobilizing the same fraction $\epsilon$ of the citizens would be more costly in district with more citizens. This assumption, though, is plausible because I assume that the increasing value of the total prize and the increasing costs of mobilization cancel as district magnitude increases. See  Footnote \ref{fn:value} for a detailed discussion.} Assuming risk neutrality, this gives the (expected) utility function for each party.
\begin{align}
u_w(\epsilon_w, \epsilon_s) &= \dfrac{(\beta \epsilon_w)^\delta}{(\beta \epsilon_w)^\delta + \epsilon_s^\delta} - \epsilon_w \\
u_s(\epsilon_w, \epsilon_s) &= \dfrac{\epsilon_s^\delta}{(\beta \epsilon_w)^\delta + \epsilon_s^\delta} - \epsilon_s
\end{align}


\subsection*{Preliminaries}

In equilibrium, each player chooses a strategy that is a function of the strength of $w$ ($\beta$) and the disproportionality of the contest ($\delta$). For many combinations of parameter values, the game has a pure strategy Nash Equilibrium in which the two parties exert the same effort. 

\begin{lemma}
\textbf{(Equal Effort)} Where a pure strategy Nash equilibrium exists, $w$ exerts the same effort as $s$ in equilibrium (and vice versa, although the amount each chooses might vary across levels of competitiveness and disproportionality). Formally, for all $\beta$ and $\delta$ where $\beta \geq (\delta - 1)^{(1/\delta)}$, $\epsilon_w = \epsilon_s$ in equilibrium.

\normalfont{Proof: See the Online Appendix.}
\end{lemma}

\begin{lemma}
\textbf{(Equilibrium Strategy)} For a sufficiently large $\beta$, the strategy profile $S^* = (s_w^*, s_s^*)$ is a Nash Equilibrium, where $s_i^*(\delta, \beta) = \epsilon_i = \delta\dfrac{\beta^\delta}{(\beta^\delta + 1)^2}$.

\normalfont{Proof: See the Online Appendix.}
\end{lemma}

\subsection*{Comparative Statics}

Next, I present comparative statics derived from Lemma 2 that describe how changes in the values of parameters change the equilibrium effort levels. I focus on three particular comparative statics: how equilibrium effort changes with competitiveness, how equilibrium effort changes with disproportionality, and how the marginal effect of competitiveness on equilibrium effort changes with disproportionality. A brief review of the intuition underlying the statement as well as an associated empirical hypothesis follow each theoretical proposition. 

\begin{proposition}
\textbf{(Competitiveness)} Where an equilibrium exists, equilibrium effort is increasing in competitiveness for all levels of disproportionality.

\normalfont{Proof: See the Online Appendix.}
\end{proposition}
This proposition has a straightforward intuition. A party has an incentive to devote its limited resources to districts in which its efforts might alter the outcome of the contest. A party should not waste its resources in noncompetitive districts in which its efforts would prove fruitless. On the other hand, as a district's contest becomes more competitive, the party should expend more resources in the district to pick up a marginal seat. Accordingly, the party should increase its mobilization effort as competitiveness increases, regardless of the level of disproportionality. 

\begin{quote}
\textsc{Competitiveness Hypothesis}: In both SMDP and PR systems, the mobilization effort by a district's parties increases as the district's competitiveness increases.
\end{quote}

The next section discusses the second theoretical proposition, its intuition, and its associated testable implication.

\begin{proposition}
\textbf{(Disproportionality)} Where an equilibrium exists, equilibrium effort is increasing in disproportionality except when both disproportionality and competitiveness are extremely low.

\normalfont{Proof: See the Online Appendix.}
\end{proposition}

\noindent This theoretical expectation runs counter to the typical thought in the literature on electoral systems, which typically assumes that, \textit{holding competitiveness constant}, the amount of mobilization does not depend on the electoral rules. Rather, the formal model suggests that disproportionality itself has a direct effect on the mobilization efforts. The intuition developed rigorously in the formal model suggests that parties have a stronger incentive to mobilize when swings of a only a few votes can shift the entire prize from one party to another. In disproportional districts, strong parties have an incentive to mobilize to protect their prize and weak parties have an incentive to win the entire prize by mobilizing slightly more voters. In proportional districts, swings of a few votes can only result in parties winning and losing a small fraction of the prize. Because of this dynamic, disproportional rules give parties a stronger incentive to mobilize than proportional rules. This leads to the following hypothesis.

\begin{quote}
\textsc{Disproportionality Hypothesis}: At any level of competitiveness in a district, the mobilization effort by the district's parties is greater in SMDP systems than in PR systems.
\end{quote}

The next section discusses the final theoretical proposition and its intuition, and presents the key empirical hypothesis of the paper.

\begin{proposition}
\textbf{(Interaction Between Competitiveness and Disproportionality)} Where an equilibrium exists, the positive marginal effect of competitiveness on equilibrium effort is increasing in disproportionality except when both disproportionality and competitiveness are extremely low.

\normalfont{Proof: See the Online Appendix.}
\end{proposition}

\noindent Proposition 3 also runs counter to most of the literature on electoral institutions. Under SMDP rules, the party that wins the election obtains the entire prize in the form of a legislative seat or seats. The intuition for Proposition 3 continues from the intuition for Proposition 2. While disproportional districts encourage mobilization by parties no matter the closeness of the race, the magnitude of this effect becomes greater as the difference in expected vote shares shrinks. As the race becomes more competitive, fewer votes are required to shift the entire prize from one party to another. No similar effect occurs in proportional districts. A swing of a few votes leads to a shift in a small share of the prize, no matter how close the race. Therefore, the disproportionality of a district should increase the effect of competitiveness. This leads to the observable implication that the marginal effect of competitiveness on the probability that citizens are contacted by a political party is greater in SMDP systems than PR systems.

\begin{quote}
\textsc{Interaction Hypothesis}: The (positive) marginal effect of a district's level of competitiveness on the mobilization effort by the district's parties is greater under SMDP rules than under PR rules.
\end{quote}

These three propositions (especially Propositions 2 and 3) suggest a much different understanding of the role of electoral institutions in influencing mobilization than is found in the current literature. The most widespread view is that proportional rules encourage mobilization by creating "nationally competitive districts." This understanding misses two important points. First, disproportional rules offer a strong, direct incentive to mobilize voters (Proposition 2). Second, while proportional district might be more competitive (though see \citealt{BlaisLago2009}), competitiveness should impact mobilization efforts relatively little. 

The previous sections presented three theoretical propositions and suggested three parallel empirical hypotheses. The next section describes the data used to test these hypotheses and defends the operationalizations of the theoretical concepts. 

\section*{Data and Measures}

To test these hypotheses, I use Module 2 of the Comparative Studies of Electoral Systems (CSES).\footnote{Modules 1 and 3 of the CSES exclude an important survey question that I use to measure mobilization.} The data set includes individual, district, and national level data from many countries around the world, relying primarily on the work of regional collaborators. Because the Module 2 data include district-level vote shares for only the top eight parties in each country, the CSES data are supplemented with district-level vote totals where necessary to obtain a more complete data set.\footnote{These additional data were compiled from Adam Carr's online election archive. See \texttt{http://psephos.adam-carr.net/}.}

To create the strongest test of the theory and the most reliable estimates, I restrict the analysis to legislative (lower-house) elections in the CSES Module 2 that fit two important criteria: choosing elections in which (1) no concurrent national-level elections are present and (2) seats are assigned using first-past-the-rules or the d'Hondt divisor system (i.e. no second-tier corrections). The five elections that fit these criteria occur in Great Britain, Canada, Finland, and Portugal (2). Because of the difficulty in measuring competitiveness in PR systems and the typical electoral complexity of these systems, any measurement error will biased the results in favor of my hypotheses. Thus, I choose those cases that are most likely to provide evidence against my theory. The Online Appendix provides a table describing the countries that are excluded and the rationale for each decision.

First and most importantly, it is difficult to conceive of and measure competitiveness in more intricate systems, such as systems with single-member districts followed by second-tier correction (often called a ``mixed-member district''). Valid and reliable measurement of district competitiveness is crucial in the empirical analysis, and it remains unclear how to construct a valid measure of competitiveness in more complicated electoral systems. The introduction of measurement error would bias the results toward confirming the hypotheses, since this error would show up primarily in the PR countries and bias the estimates toward zero. Fewer cases make the test more difficult because larger effects are needed to overcome the additional uncertainty and the introduction of measurement error would bias coefficient estimates in PR elections in the direction predicted by the hypotheses. Fortunately, recent work \citep{GrofmanSelb2009} presents a compelling measure of competitiveness that is comparable across d'Hondt systems, of which SMDP is a subset. 

Second, the theory applies to a single contest for a set of legislative seats. In this situation, it is crucial to carefully select comparable cases. In particular, \cite{Achen2005} argues that large data sets are inferior when they include theoretically extraneous cases and that it is preferable to restrict the analysis a small, carefully chosen, homogenous subset (see \citealt{Gowa1999} and \citealt{Miller1999} for examples). Achen writes that ``in most of our empirical analyses, some groups of observations should typically be discarded to create a meaningful sample with a unified causal structure'' (446).\footnote{Achen continues: \begin{quote} Contrary to the received wisdom, it is not the``too small'' regressions on modest subsamples with accompanying plots that should be under suspicion. Instead, the big analyses that use all the observations and have a dozen control variables are the ones that should be met with incredulity. \end{quote}} If concurrent elections are present, such as an upper-house or presidential election, it is unclear how this would affect the incentives of parties. It might be that the efforts of presidential candidates overwhelm the efforts of legislative candidates, or the two might balance out. Also, given the measure of mobilization discussed below, it is impossible to distinguish which race is generating the mobilization effort. In order to generate the most accurate estimates, it is important to be able to assign the observed mobilization effort to a particular electoral contest.

\subsection*{Mobilization}

The empirical hypotheses presented above, following \cite{KarpBanducciBowler2007}, rely on the probability of being contacted by a political party as a measure of mobilization efforts in a district.Other scholars use aggregate measures of mobilization to test whether parties make a greater mobilization effort in more competitive districts, but contact data offer several advantages over this more common variable. For example, \cite{CoxMunger1989} use campaign expenditures as an indicator of mobilization efforts. No such measure exists in the CSES data. As a strong alternative, this analysis relies on an individual-level variable that asks respondents, ``During the last campaign did a candidate or anyone from a political party contact you to persuade you to vote for them?" In some sense, this measure taps the concept of mobilization more directly than an indirect aggregate measure, such as campaign expenditures. Most countries do not make campaign expenditures publicly available, but many surveys inquire about contact with political parties. Because of this, using the individual-level measure allows for testing hypotheses in more general settings. Also, the contact measure only captures one of the many potential forms of mobilization. Parties might run ads on television or in newspapers or conduct rallies or other political events. All these activities constitute mobilization, but the self-reported contact measure misses each.\footnote{Also, notice that the survey question does not ask about contact from activists, who face the same incentives as political parties and candidates.} This criticism, while valid, does not pose as large a problem as it might first appear. Previous research on party mobilization \citep{GerberGreen2000, HuckfeldtSprague1992} shows that while parties engage in other forms of mobilization, more personal forms of contact have larger effects on voter participation. Thus, self-reported contact might not fully capture party mobilization efforts, but it captures an important and effective type of mobilization. Thus, the contact variable, while lacking the breadth of the campaign expenditure measure, more directly taps the concept of mobilization, focuses on an important and effective form of mobilization, and allows testing hypotheses in more diverse settings.

\subsection*{District Competitiveness}

Competitiveness varies across districts. While researchers can easily compare the competitiveness of one district to another in a SMDP system, comparing the competitiveness of districts in PR systems seems more difficult. On top of this, it seems even more difficult to compare district competitiveness across electoral formulas. Prior research dodges this problem by assuming that PR rules create competition over each seat in every district. Nonetheless, scholars can and should measure district competitiveness comparably across districts and across countries.

Scholars typically measure competitiveness in SMDP districts by taking the difference in vote share between the first and second place finishers. This fits well with most conceptions of competitiveness. However, few prior studies actually measure competitiveness at the district level in PR systems. Recent papers by \cite{BlaisLago2009}, \cite{GrofmanSelb2009}, and \cite{Selb2009} argue that competitiveness varies at the district level in PR systems and that research must take this variation into account when considering the effect of competitiveness. Prior studies fail to appropriately capture this variation. For example, \cite{Franklin2004} assumes that all PR districts feature competitiveness similar to a tie under SMDP rules. Most scholars would probably agree that PR rules do not create perfectly competitive districts, yet some continue to assume otherwise in their analyses. Recent work liberates scholars from dubious assumptions about competitiveness under PR rules.\footnote{\cite{BlaisLago2009} measure competitiveness by ``the minimal number of additional votes required, under existing rules, for any party to win one additional seat." While this measure improves upon prior assumptions about competitiveness in PR districts, room for improvement still exists. This measure seems to overstate the competitiveness in PR districts because it only provides information about the closeness of the closest race. This works in SMDP systems because summarizing the closeness of the closest races summarizes the closeness of the district. However, as the district magnitude increases, some parties have closer races than others. The measure recommended by \cite{BlaisLago2009} only captures the competitiveness of the closest contest and overstates the competitiveness of the district. \cite{Selb2009} offers a different solution and considers only the contest for the final seat, but it has similar shortcomings. The race for the final seat tells us something about the competitiveness of the district, but it does not adequately summarize the competitiveness for each seat. Even if a district has a close race for the final seat, parties not hotly contesting the final seat might win or lose many votes without winning or losing any seats. On the other hand, if the district does not produce a close race for the final seat, it cannot create competition over the other seats. In other words, the competitiveness for the final seat does not tell us much about the competitiveness of the district as a whole, but it does provide a rough upper bound.}

\cite{GrofmanSelb2009} offer an excellent\footnote{While other measures, such as the measure offered by Blais and Lago (2009), consider only the incentives of the two parties closest to winning and losing, respectively, the final seat, the measure offered by Grofman and Selb (2009) considers the incentives of all the competing in the contest. } but less general measure of district competitiveness applicable specifically to d'Hondt systems. The analysis presented below uses this measure.\footnote{The results were also replicated using the measure developed by \cite{BlaisLago2009}. There are no substantive differences in the results.} Informally, the measure takes account of each party's incentive to mobilize voters, determined by the number of votes that guarantees a party another seat or might cause the party to lose a seat. The measure weights the larger of these two incentives by the party's vote share to find the competitiveness in a district. 

To build their measure, \cite{GrofmanSelb2009} first notice that a party must earn a certain number of votes in order to guarantee itself another seat. Equation \ref{gain_eqn} gives the number of additional votes a party must win to guarantee itself an additional seat.
\begin{equation}\label{gain_eqn}x^G_i = [(s_i + 1) / (m + 1)] - v_i 
\end{equation}
if $s_i < m$ and $x^G_i \leq T^E$, where $s_i$ represents the number of seats won by party $i$, $m$ represents the district magnitude, $v_i$ represents party $i$'s vote share, and $(s_i + 1)/(m + 1)$ represents the threshold of exclusion $T^E$ for the $(s + 1)$th seat. Second, they notice that in order to possibly lose a seat, a party must lose a certain number of votes. Equation \ref{loss_eqn} gives the number of votes a party must lose to possibly lose a seat.
\begin{equation}\label{loss_eqn}
x^L_i = (-s_iv_j + s_jv_i + v_i)/(s_i + s_j + 1)
\end{equation}
if $s_i >0$, where party $j$ finishes second in the contest for $i$'s final seat.\footnote{In the published version of their paper, Equation \ref{loss_eqn} contains an error. The second $s_i$ in Equation 5 of their paper should be subscripted with $j$. The error is corrected in Equation \ref{loss_eqn} of this paper.} To determine the incentive for party $i$ to mobilize, \cite{GrofmanSelb2009} assume that party $i$ responds to the larger of the two incentives, $x^G_i$ and $x^L_i$. By finding the larger incentive and standardizing it by $T^E$, they find the incentive for party $i$ to mobilize. Equation \ref{compi_eqn} gives the incentive for party $i$ to mobilize in a district.
\begin{equation}\label{compi_eqn}
c_i = \text{max}[(T^E - x^G_i), (T^E - x^L_i)]/T^E \text{ ,}
\end{equation}
where $T^E$ equals $1/(m +1)$. \cite{GrofmanSelb2009} simply take the average of $c_i$ weighted by $v_i$ to combine each incentive $c_i$ to mobilize. Equation \ref{comp_eqn} gives the total competitiveness in a district, $C$.
\begin{equation}\label{comp_eqn}
C = \displaystyle\sum_{i=1}^n v_ic_i
\end{equation}
$C$ describes the total competitiveness of the district. This measure improves upon other measures by estimating closeness for each party in each district and weighting by the vote share for that party. This provides a conceptually coherent method to summarize the total competitiveness of the district. Figure \ref{fig:CbyC} presents histograms showing the distribution of competitiveness for each election included in the analysis using Grofman and Selb's \citeyearpar{GrofmanSelb2009} measure.

\subsection*{Disproportionality}

To capture the concept of disproportionality, the analysis uses a dummy variable with SMDP systems (Canada and Great Britain) coded as one and PR systems (Finland and Portugal) coded as zero. Other scholars rely on finer measures of disproportionality, such as the average district magnitude, but that does not seem useful for the analysis. Most of the important variation in disproportionality occurs when district magnitude is close to one \citep{GrofmanSelb2011}. A large change in the proportionality of the district occurs when a district's magnitude changes from one to two. However a much smaller change occurs when a district's magnitude increases from three to four or nine to ten. This suggest a strong non-linear relationship that is more appropriately modeled using dummy variables. Future work might use the variation in district magnitude to gain further empirical leverage, but for the purposes of this analysis, the key variation occurs across systems types, not within \citep{ TaageperaShugart1989, Taagepera2007,GrofmanSelb2011}.\footnote{Portugal offers substantial variation in district magnitude, but unfortunately the smallest district has magnitude three. The most significant substantive change in proportionality occurs in the change from a single-member district to a two-member district. It is quite easy to imagine that a 3-seat district has outcomes as proportional as a district with 10 seats. On the other hand, a single-member district is almost always less proportional than a two-member district and virtually guaranteed to be less proportional than a 3-member district. Models of party competition offer some insight into this as well. Assuming $m-1$ parties are competing in the contest, then up to 50\% of voters might vote for the loser in single-member districts. As the district magnitude increases, the percent of voters who might vote for the loser drops to 33\%, 25\%, and 20\%. Thus, once district magnitude rises to about 3 or 4, the proportionality of the districts changes very little. This theoretical intuition that $m$ is not a good measure of disproportionality once $m > 1$ argument is supported empirically by national \citep{Taagepera2007, TaageperaShugart1989} and district-level evidence \citep{GrofmanSelb2011}. I did, however, estimate the model with district magnitude as a measure of disproportionality. As expected, the results are ambiguous.}

\section*{Empirical Model}

When considering possible models to fit to these data, the nested structure of the data becomes immediately apparent. This nesting creates dependence among the observations, which requires post-estimation corrections or a model more complicated than the standard logit or probit. Parties behave differently across countries. Norms constrain political parties and vary across countries and cultures. Respondents have similar, unobserved characteristics to others in their district or country. Competition might have a homogeneous effect within countries and a heterogeneous effect across countries \citep{Western1998}. Possible solutions include modeling the dependence directly with a multilevel model or using more conventional techniques and correcting standard errors after estimation.

A multilevel modeling approach offers several advantages. First, it is necessary to estimate a separate intercept for each district, but some districts feature only a few respondents (and sometimes only one!). This requires a multilevel model to estimate a group variance parameter to smooth the district intercepts. As the number of individuals in a group becomes large, the amount of smoothing decreases and traditional approaches tend to give similar estimates. Finally, directly modeling variation and heterogeneity that might not directly interest the analyst leads to a cleaner final analysis, easier evaluation of the model, and clearer understandings of the model fit \citep{Gelman2005, Gelman2006}.

Software for estimating multilevel models has improved over the last five years and the applicability of the multilevel approach has increased dramatically with the widespread availability of the Cross-National Studies of Electoral Systems data and other cross-national surveys such as the Eurobarometer and the World Values Survey. To estimate the models below, Bayesian MCMC algorithms generate samples from the posterior distributions of interest \citep{Gill2008, Gelmanetal2004, GelmanHill2007}. Weakly informative priors \cite{Gelman2006a} allow the data to almost completely drive the inferences while still harnessing the power and flexibility of MCMC algorithms \citep{Jackman2000, Jackman2004}.\footnote{As a prior for the fixed effects, I use a normal distribution with mean equal to zero and variance equal to 1000, which practically serves as  a flat prior distribution. For the correlation parameter, I use a uniform distribution from -1 to 1. For the standard deviations of the random effects, I use a half-Cauchy with the scale parameter equal to five. The data swamp the prior for all parameters except the election-level random effects and alternative flat and informative  priors for the standard deviations barely affect the inferences and do not affect the substantive conclusion. See the Online Appendix for more details about the choice of prior for the standard deviation of the random effects}

The individual-level model is a simple varying intercept model that includes individual-level covariates. Given the small number of respondents from some districts, particularly in Britain and Canada, it is important to include individual-level covariates that  predict the whether a person is likely to be contacted. Because several of the variable have several missing values, I multiply imputed the data set \citep{Rubin1987, Kingetal2001}.\footnote{Using listwise deletion yields similar results, and the substantive conclusions are identical.} The probability that individual $i$ in district $j$ in election $k$ is contacted by a political party is modeled as
\begin{equation}
\text{Pr}(Contacted_{i} = 1) = \text{logit}^{-1}(\alpha_{jk} + {\bf X}_i\beta) \text{ ,}
\end{equation}\vspace{3mm}
where $\alpha_{jk}$ represents an intercept that varies across districts and elections, ${\bf X}_i$ represents a matrix of individual-level covariates, excluding the constant, and $\beta$ represents a vector of non-varying coefficients. In particular,
\begin{align}
{\bf X}_i\beta & = \beta_{1}\text{Age}_i + \beta_{2}\text{Male}_i + \beta_{3}\text{Education}_i \nonumber\\
& + \beta_{4}\text{Union Member}_i + \beta_{5}\text{Household Income}_i\\
& + \beta_{6}\text{Urban}_i + \beta_{7}\text{Close to a Party}_i \text{ .} \nonumber
\end{align}
\noindent I code the variables as follows: age is simply the age of the respondent, male is an indicator variable for male respondents, education is a seven-point ordered-categorical variable, household income is an ordered categorical variable that indicates the respondent's income quintile, urban is an indicator variable for respondents from large urban areas, and close to party is an indicator variable for respondents who reported feeling close to a political party.

Notice that while the empirical model occurs at the level of survey respondents, the primary parameter of interest is the district-level parameter $\alpha_{jk}$, which captures the level of mobilization in the district. Consistent with the theory discussed above, the varying-intercept $\alpha_{jk}$ is modeled as a function of district competitiveness. Importantly, the model allows competitiveness to have an effect in both PR and SMDP systems. Previous research, such as \cite{Franklin2004} and \cite{KarpBanducciBowler2007}, estimate models that assume (1) that competitiveness does not vary across districts in PR systems or (2) that the effect of competitiveness on the probability of being contacted is zero.\footnote{\cite{GrofmanSelb2009}, \cite{Selb2009}, and \cite{BlaisLago2009} argue that competitiveness can vary across districts in PR systems and offer evidence of meaningful variation.} The theoretical argument suggests that competitiveness should have an effect in PR systems so the model allows it to vary. This interaction is crucial to test the theory. As each district becomes more competitive, citizens in the district should be more likely to receive contact from a political party. In particular, 
\begin{equation}
\alpha_{jk} \sim N(\gamma_{0k} + \gamma_{1k}Competitiveness_j, \sigma^2_{\alpha}) \text{ for } j = 1, 2,..., J \text{ ,}
\end{equation}
where $J$ is the number of districts included in the analysis.

However, the theory suggests that district competitiveness has a larger effect in systems with disproportional rules. Because of this, the coefficient for competitiveness is modeled as a function of disproportionality. When allowing a coefficient to vary, it almost always makes sense to allow the intercept to vary as well and doing so is consistent with other literature in political science on interaction terms \citep{Friedrich1982, BramborClarkGolder2006}. The intercept is modeled as a function of the electoral rules. Modeling the intercept also allows a formal test of the Disproportionality Hypothesis. The varying intercept and slope are allowed to correlate (see \citealt{GelmanHill2007}, esp. ch. 13). In particular, the election-level coefficient and intercept are modeled as
\begin{equation}
\begin{pmatrix} \gamma_{0k} \\ \gamma_{1k} \end{pmatrix} 
\sim 
{\large} N\begin{pmatrix}  \begin{pmatrix} \mu_{\gamma_{0}} \\ \mu_{\gamma_{1}} \end{pmatrix}, 
\begin{pmatrix}         \sigma^2_{\gamma_{0}}                                           &                 \rho\sigma_{\gamma_{0}}\sigma_{\gamma_{1}} \\
                                 \rho\sigma_{\gamma_{0}}\sigma_{\gamma_{1}}         &                \sigma^2_{\gamma_{1}}
\end{pmatrix}\end{pmatrix} \text{ , for }k = 1, 2, ..., K \text{ ,}
\end{equation}
where 
\begin{equation}\label{eqn:disint_modeled}
\mu_{\gamma_{0}} = \delta_{00} + \delta_{01}Disproportional.Rules_k \text{  ,}
\end{equation}
\begin{equation}\label{eqn:comp_modeled}
\mu_{\gamma_{1}} = \delta_{10} + \delta_{11}Disproportional.Rules_k \text{ ,}
\end{equation}          
and $K$ is the number of elections included in the analysis.\footnote{Importantly, this structure allows estimating a separate intercept and slope (for competitiveness) in each election. There are many reasons to expect small differences across countries, such as variation in the level of intra-party competition. The model allows these differences to emerge.}
                                                                                
\section*{Results}

To assess the convergence of the Gibbs sampling algorithm, I calculated $\hat{R}$ statistics \citep{GelmanRubin1992} for each parameter of the model. For a single imputed data sets, I ran three MCMC chains until the $\hat{R}$ statistics for each parameter dropped below 1.01. The model appears to converge after about 2,000 draws. Using this sense of how many iterations  the Markov chain takes to converge, I use a conservative 5,000 iteration burn-in period, followed by 10,000 simulations with each of five multiply imputed data sets. I combine the simulations from these five chains to obtain the posterior simulations, which contain sufficient information to test the three hypotheses suggested by the theory.\footnote{\label{fn:robust}In addition to the model presented above, I estimated and evaluated several other models using likelihood ratio tests, the deviance information criterion, and cross-validation. The results presented here are robust to different modeling choices, including hierarchical and non-hierarchical structures, inclusion and exclusion of individual-level covariates, and multiple imputation and listwise deletion. One potential confound of concern is the party system and specifically the effective number of parties. The number of parties is known to covary with electoral rules and can reasonably be thought to influence the rate of contacting. However, including this control does not affect the results. See the Online Appendix for the details of several robustness checks.}Rather than present and discuss the coefficients directly, I discuss changes in probability of being contacted as implied by the model \citep{KingTomzWhittenburg2000, BerryDeMerittEsarey2010} and rely on 90\% Bayesian credible intervals that correspond to a one-sided test with $\alpha=0.05$. To compute predicted probabilities, I set all individual-level covariates at their medians.

\subsection*{The Effect of Competitiveness}

According to the Competitiveness Hypothesis, the theory suggests that the probability of receiving contact from a political party increases as district competitiveness increases, regardless of the disproportionality of the electoral rules. The left panel of Figure \ref{FullPrProb} shows little change in the predicted probability of contact as district competitiveness increases in PR systems. Confirming what Figure \ref{FullPrProb} suggests, the left panel of Figure \ref{mfx} shows that the marginal effect of competitiveness is nearly zero in PR systems at all levels of competitiveness. The data offer little support for the Competitiveness Hypothesis in PR systems, because the marginal effect of competitiveness is always close to zero and the credible intervals substantially overlap zero.\footnote{\label{fn:noeffect}However, it is important to note that the results do not suggest that competitiveness has no effect in PR systems. Indeed, the data are consistent with substantively large effects \citep{Rainey2014}. However, a true relationship that is inconsistent with the theory (i.e. competitiveness has a negative effect on mobilization efforts) cannot be confidently ruled out. The data are consistent with relationships suggested by the theory and other relationships as well. These ambiguous results are not completely unexpected, though, due to the likely small effect of competitiveness in PR systems (the theory does suggest a small effect) and the small sample size. The important empirical result, that competitiveness has a larger effect in SMDP systems than PR systems, is supported and discussed below.}

The Competitiveness Hypothesis also suggests that district competitiveness should have a positive effect in SMDP systems. The right panel of Figure \ref{FullPrProb} shows that the predicted probability of contact increases dramatically across the range of district competitiveness in SMDP systems, suggesting that the hypothesis is correct. The right panel of Figure \ref{mfx} confirms this, showing clearly that the marginal effect of competitiveness on the probability of receiving contact is increasing for all levels of competitiveness. Therefore, the empirical model provides strong support for the Competitiveness Hypothesis in systems with disproportional rules. Competitiveness exerts a large positive effect on party mobilization under disproportional rules.

\begin{figure}[h]
\centering
\includegraphics[scale = .7]{figs/prob.pdf}
\caption{This figure contrasts the effects of district competitiveness on the predicted probability of receiving contact from a political party under different electoral formulas. The solid line indicates the estimate of the probability of receiving contact and the dotted lines show the 90\% Bayesian credible interval around that estimate.The histograms in the background show the distribution of competitiveness under PR and SMDP electoral rules, respectively. The figure shows that the probability of receiving contact shrinks near zero under both formulas in non-competitive districts. However, the probability of receiving contact increases drastically under SMDP rules as competitiveness intensifies, but does not increase as much under PR rules. Compare these figures to the theoretical expectations given in the comparative statics. The histograms in the background show the distribution of district competitiveness in each type of system.}\label{FullPrProb}
\end{figure}

\begin{figure}[h]
\centering
\includegraphics[scale = .7]{figs/me_competitiveness.pdf}
\caption{This figure shows the marginal effects of competitiveness in PR and SMDP systems. The solid line indicates the estimate of the marginal effect and the dotted lines show the 90\% Bayesian credible interval around that estimate. The Competitiveness Hypothesis suggests that competitiveness should have a positive effect on the probability of receiving contact in both PR and SMDP systems for all levels of competitiveness. This figure shows little support for the Competitiveness Hypothesis in PR systems, but strong support in SMDP systems. The histograms in the background show the distribution of district competitiveness in each type of system.}\label{mfx}
\end{figure}
\newpage

\subsection*{The Effect of Disproportionality}

The Disproportionality Hypothesis suggests that the effect of disproportionality should be positive for all values of competitiveness, regardless of the electoral rules. The empirical model presented above supports this hypothesis. Again, because the effect of competitiveness and the district-level intercept are modeled as a function of disproportional rules, disproportionality has an effect that varies across values of competitiveness. Comparing the left and right panels of Figure \ref{FullPrProb} suggests that this hypothesis only holds for high values of competitiveness, since the predicted probabilities are similar for low values of competitiveness. Figure \ref{FD} provides the difference in the predicted probability of being contacted by a political party in SMDP and PR systems as competitiveness varies. This figure shows that disproportional rules have a positive, statistically significant effect on the difference in the predicted probabilities of being contacted only for the higher values of competitiveness. This offers only partial support for the Disproportionality Hypothesis, but provides strong evidence that disproportionality has a large, positive effect on mobilization efforts in more competitive districts.

\begin{figure}[h]
\centering
\includegraphics[scale = .7]{figs/fd_smdp.pdf}
\caption{This figures shows how the predicted probability of an individual receiving contact from a political party changes when that individual ``moves'' from a system that features PR rules to a system with SMDP rules. The solid line indicates the estimate of the difference and the dotted lines show the 90\% Bayesian credible interval around that estimate. The figure shows that for the low values of competitiveness observed in the data, no difference exists between the systems. However, as competitiveness intensifies, SMDP rules increase the probability of receiving contact from a political party relative to PR rules. The histogram in the background shows the distribution of competitiveness across system types. I omit the histogram showing the distribution of the data because this plot mixes the two electoral systems. Refer to Figures \ref{FullPrProb}, \ref{mfx}, or \ref{fig:CbyC} for the distribution of the data.}\label{FD}
\end{figure}

Overall, the model combined with the additional analysis offers some support for the Disproportionality Hypothesis. More citizens are contacted by political parties in disproportional systems. However, the model suggests that the evidence for this difference is strong only in more competitive districts.

\subsection*{The Interaction Between Competitiveness and Disproportionality}

So far, the analysis suggests that district competitiveness has a statistically significant effect on the probability of contact in SMDP systems but not in PR systems; however, this does not automatically suggest that the two effects are statistically different \citep{GelmanStern2006}. The Interaction Hypothesis predicts significant differences between the marginal effect of district competitiveness under different electoral rules. While Figure \ref{mfx} suggests that the marginal effect of competitiveness in greater in SMDP systems, the large credible intervals make accepting the Interaction Hypothesis quite difficult. Figure \ref{mfxdif} offers an explicit test of the prediction by plotting the difference between the marginal effect of competitiveness in SMDP and PR systems as competitiveness varies. This figure shows that the marginal effect of competitiveness is greater in SMDP elections for all levels of competitiveness. This test offers strong support for the Interaction Hypothesis. The marginal effect of district competitiveness in SMDP systems is significantly larger than in PR systems. 

\begin{figure}[h]
\centering
\includegraphics[scale = .7]{figs/interaction.pdf}
\caption{This figures shows the difference between the marginal effects of competitiveness in SMDP and PR systems as competitiveness varies. The solid line indicates the estimate of the difference in the marginal effects and the dotted lines show the 90\% Bayesian credible interval around that estimate. Consistent with the Interaction Hypothesis, this figure shows that the marginal effect of competitiveness is always greater in SMDP systems. The histogram in the background shows the distribution of competitiveness across system types. I omit the histogram showing the distribution of the data because this plot mixes the two electoral systems. Refer to Figures \ref{FullPrProb}, \ref{mfx}, or \ref{fig:CbyC} for the distribution of the data.}\label{mfxdif}
\end{figure}

\subsection*{Considering the Individual Countries}

The analysis above focuses on the averages across countries. The predicted probabilities and marginal effects are for a hypothetical, ``average'' election. However, the observed elections also provide support for the conclusions from the larger model. Figure \ref{fig:CbyC} shows the predicted probability of receiving contact from a political party as competitiveness varies in each country included in the analysis. The results are consistent and robust across countries. As suggested by my theory, the increase in the predicted probability as competitiveness increases is larger in the SMDP systems included in the analysis (Canada and Great Britain) and smaller in the PR countries (Finland and Portugal). The difference is stark and shows the ability of the model to explain party strategies under different electoral rules.

\begin{figure}[h]
\centering
\includegraphics[scale = .7]{figs/CbyC.pdf}
\caption{This figures shows how the predicted probability of an individual receiving contact from a political party changes as competitiveness increases across the countries included in the analysis. The solid line indicates the estimated probability and the dotted lines show the 90\% Bayesian credible interval around that estimate The figure shows that the results in Figures \ref{FullPrProb} and \ref{FD} are robust. The two SMDP elections in the data (Canada and Great Britain) show large increases in the probability of receiving contact as competitiveness increases, while the three PR elections (Finland and Portugal) show no increase. This offers strong support for the Interaction Hypothesis. The histograms in the background show the distribution of competitiveness in each country.}\label{fig:CbyC}
\end{figure}

In summary, the data are supportive of the key intuitions of the theory, supporting the Interaction Hypothesis and partially supporting the secondary Competitiveness Hypothesis and Disproportionality Hypothesis. First, district competitiveness has a significant effect on the likelihood of being contacted, but only in systems with disproportional rules. Competitiveness has no appreciable effect in systems with proportional rules. Second, disproportional rules have a positive effect on the likelihood of being contacted, but only for highly competitive districts. Finally and most importantly, the effect of competitiveness becomes significantly stronger under disproportional rules relative to proportional rules.

The findings here suggest that previous research misses an important effect of disproportionality. While previous work suggests that disproportionality \textit{discourages} mobilization, I present a theory suggesting that disproportionality  \textit{encourages} mobilization by exaggerating the effect of competitiveness. Empirically, I find that disproportionality actually boosts the likelihood of receiving contact from a political party in competitive districts and that competitiveness always has a larger positive effect on mobilization in SMDP systems. While this result says nothing about the effectiveness of the contact (i.e. whether the citizen responds by turning out to vote), it does show that parties in competitive SMDP districts make a much stronger effort than parties in similar PR districts, contrary to the claims of most previous research. Further, parties' mobilization efforts in the \textit{most competitive} PR districts appear similar to those in the \textit{least competitive} SMDP districts. 

\section*{Discussion}

This section offers a brief discussion of the relationship between the theoretical and empirical findings presented above and other scholarship. I explain where my findings are compatible with previous research and how it differs. First, previous literature claims that PR rules increase district competitiveness on average \citep{Powell1986, Jackman1987, BlaisDobrzynska1998, BlaisAarts2006, Blais2006}. The results presented here are entirely compatible with this claim. The data shown in Figure \ref{fig:CbyC} show that within the data used for the analysis, SMDP systems feature more ``non-competitive'' districts than PR systems. However, the analysis also suggests that the emphasis placed on the observed difference in competitiveness across systems is unwarranted. Both the theory and empirical model suggest that competitiveness has a much smaller impact on mobilization efforts than previously thought in PR systems. Indeed, my data suggest that PR rules nearly eliminate any mobilizing incentives of competitiveness. 

Second, more recent literature motivated by formal models \citep{Cox1999, Selb2009} argues that both competitiveness and turnout (and hence mobilization) are more variable in SMDP systems. This again is compatible with and supported by the theoretical and empirical results presented here. Indeed, the predicted probabilities I present in Figure \ref{fig:CbyC} show that the predicted probability of being contacted in SMDP systems varies substantially across competitiveness. Much less variation exists in PR systems. However, this is not due to PR rules creating an incentive to mobilize everywhere, as much of the literature suggests. Rather, PR rules create no strong incentive for parties to mobilize anywhere. Indeed, mobilization efforts in the least competitive SMDP districts are comparable to those in the most competitive PR districts.

The claims of this paper are vulnerable to several alternative explanations that future research should investigate more carefully. First, one might suggest that competitiveness remains constant across elections in SMDP systems but changes unpredictably in PR systems. Because competitiveness is not predictable in PR systems, parties cannot act strategically. While this might explain the lack of an increase in the predicted probability of contact as competitiveness increases, it does not explain why contact rates are so low in PR systems. Future work should also investigate whether parties in PR and SMDP systems use different modes of campaigning to attract voters. It could be that parties use advertising strategies in PR systems and a canvassing approach in SMDP systems. \cite{KarpBowlerBanducci2003} offer some tentative evidence that this might occur based on the 1999 European Parliamentary Elections and \cite{ZittelGschwend2008} argue that campaign strategies might vary with electoral rules based on the 2005 German elections.

This work also lays the groundwork for several important extensions. First, similar analyses should be conducted in and across more countries and under a wider variety of electoral rules. Doing so will require extension of the measure of district competitiveness introduced by \cite{GrofmanSelb2009} to other electoral formulas. Alternatively, future work might sacrifice validity for a less precise but more widely applicable measure. Second, future theoretical work might focus on parties' incentives under simultaneous national elections, such as elections to an upper and lower house, presidential and legislative elections, and so on. Future work might also focus on mixed electoral systems, building on the foundation laid in this paper, as well as previous work by \cite{FerraraHerron2005} and \cite{Ferrara2006} on strategic entry.

\section*{Conclusion}

The resolution to the debate over whether PR rules cause more mobilization efforts and higher turnout has important implications for representative democracy. As noted by many previous studies \citep{WolfingerRosenstone1980, RosenstoneHansen1993, BradyVerbaSchlozman1995}, wealthier, more educated, and higher SES citizens turn out at a greater rate than other citizens. Because elected officials have an incentive to respond to voters rather than the citizens as a whole \citep{Downs1957}, the resulting policies reflect the interests of only some citizens. While scholars disagree over the severity of this problem \citep{BerelsonLazarsfeldMcPhee1954, Lijphart1997, Teixeira1992}, most agree that low turnout poses an obstacle to an ideal democracy. Indeed, Arend Lijphart (1997) calls unequal participation ``democracy's unresolved dilemma,'' and suggests PR electoral institutions as a resolution.

Further, \cite{Sniderman2000} points out that parties play an important role in structuring the political world, allowing relatively uninformed voters to make sense of it. Political scientists know a great deal about how many parties are likely to emerge in a political system (e.g. \citealt{Cox1997, ChhibberKollman1998, ClarkGolder2006}) and where these parties are likely to position in the ideological space (e.g. \citealt{Cox1990, KollmanMillerPage1992, AdamsMerrillGrofman2005}), but political scientists know relatively less about what rules give parties an incentive to mobilize voters, making political participation less costly and providing voters with the information necessary to make good choices \citep{Downs1957}.


This paper makes an important contribution to this debate, suggesting that previous work misses the important impact of disproportionality itself on incentives to mobilize. I argue that disproportionality itself actually encourages mobilization, since increases in effort can lead to even larger increases in seat share. I further argue that while competitiveness might be greater in PR systems, the impact of this additional competition is minimal. Thus, PR systems do not create any strong incentives to mobilize.

Political scientists have argued that one benefit of PR rules is increased voter participation (e.g. \citealt{Lijphart1997}) and other scholars have argued that this occurs because of ``nationally competitive districts'' (e.g. \citealt{Powell1986}). However, \cite{BlaisAarts2006} argue that political scientists should not be confident in these claims until a better explanation of the phenomenon is developed. The results presented in this paper affirm Blais and Aarts' skepticism by offering theoretical and empirical evidence that SMDP rules cause parties to mobilize more, not fewer, voters, especially as district competitiveness increases. Indeed, it seems that proportional rules offer parties no strong incentive to mobilize anywhere.

\singlespace 
\normalsize
\singlespace
\bibliographystyle{apsr_fs}
\bibliography{bibliography}


\begin{appendix}

\newpage


\begin{center}
{\LARGE Online Appendix}\\\vspace{2mm}
{for}\\\vspace{2mm}
{\Large Strategic Mobilization}\\\vspace{2mm}
{\large Why Proportional Representation Decreases Voter Mobilization}\\
\end{center}

%\small

\tableofcontents

\section{Proofs}

\subsection{Proof of Lemma 1 (Equal Effort)}

In order for the strategy profile $ S^* = (s_w^*, s_s^*)$ to be a Nash equilibrium, the first-order conditions, $\dfrac{\partial u_w}{\partial \epsilon_w} = 0$ and $\dfrac{\partial u_s}{\partial \epsilon_s} = 0$, and second order conditions, $\dfrac{\partial ^2u_w}{\partial ^2\epsilon_w} < 0$ and $\dfrac{\partial ^2u_s}{\partial ^2\epsilon_s} < 0$, must hold. Since the utility functions generate a discontinuous payoff space, I must also show that neither party has an incentive to defect from the proposed equilibrium to zero effort.

\textbf{Remark:} The algebra below assumes that $\epsilon_i \neq 0$ in equilibrium. No Nash equilibrium exists in which any player chooses zero effort. If both players choose zero effort, then both have an incentive to defect to an arbitrarily small amount of effort to win the whole prize. If only one player chooses to play zero effort, then the other has an incentive to play as close to zero as possible. Because no positive real number closest to zero exists, no Nash equilibrium exists where one player plays zero.

Given that $\epsilon_i \neq 0$ in equilibrium, Equation \ref{fo_A} gives the first order conditions for $u_w$.
\begin{align}
\dfrac{\partial u_w}{\partial \epsilon_w} = &\dfrac{\delta \beta^\delta \epsilon_w^{\delta-1} \epsilon_s^\delta}{[(\beta\epsilon_w)^\delta + \epsilon_s^\delta]^2} -1 = 0 \nonumber\\
&\dfrac{\delta \beta^\delta \epsilon_w^{\delta-1} \epsilon_s^\delta}{[(\beta\epsilon_w)^\delta + \epsilon_s^\delta]^2} = 1\label{fo_A}
\end{align}
Equation \ref{fo_B} presents the similar first order condition for $s$.
\begin{align}
\dfrac{\partial u_s}{\partial \epsilon_s} = &\dfrac{\delta \beta^\delta \epsilon_w^{\delta} \epsilon_s^{\delta - 1}}{[(\beta\epsilon_w)^\delta + \epsilon_s^\delta]^2} -1 = 0 \nonumber\\
&\dfrac{\delta \beta^\delta \epsilon_w^{\delta} \epsilon_s^{\delta - 1}}{[(\beta\epsilon_w)^\delta + \epsilon_s^\delta]^2} = 1\label{fo_B}
\end{align}
Since both the left hand side of Equation \ref{fo_A} and \ref{fo_B} equal one, setting one equal to the other shows that the first-order conditions hold if and only if the parties exert equal effort.
\begin{align}
\dfrac{\delta \beta^\delta \epsilon_w^{\delta-1} \epsilon_s^\delta}{[(\beta\epsilon_w)^{\delta} + \epsilon_s^\delta]^2} &= 
\dfrac{\delta \beta^\delta \epsilon_w^{\delta} \epsilon_s^{\delta - 1}}{[(\beta\epsilon_w)^\delta + \epsilon_s^\delta]^2}\nonumber\\
\epsilon_w &= \epsilon_s\nonumber
\end{align}
Given that parties must exert equal effort in any pure strategy Nash equilibrium, the second order conditions  confirm that such a strategy maximizes (rather than minimizes) the local utility. Equation \ref{eq:soc_w} gives the second-order condition for $u_w$.
\begin{align}
\label{eq:soc_w} \dfrac{\partial ^2u_w}{\partial ^2\epsilon_w} = &\dfrac{\delta (\delta - 1) \beta \epsilon_w^{\delta-2} \epsilon_s^{\delta}}{(\beta\epsilon_w^\delta + \epsilon_s^\delta)^2} -\dfrac{2\delta^2 \beta \epsilon_w^{2\delta - 2} \epsilon_s^{\delta}}{(\beta\epsilon_w^\delta + \epsilon_s^\delta)^3} < 0
 \end{align}
After applying Lemma 1 and noting that $\delta$ and $\beta$ are restricted to positive values, algebraic simplification shows that Equation \ref{eq:soc_w} holds if and only if $\beta > \left (\dfrac{\delta - 1}{\delta + 1} \right ) ^{\dfrac{1}{\delta}}$. Next, I show that the second-order condition holds for $s$.
\begin{align}
\label{eq:soc_s} \dfrac{\partial ^2u_s}{\partial ^2\epsilon_s} = &\dfrac{\delta (\delta - 1) \beta^\delta \epsilon_w^{\delta} \epsilon_s^{\delta-2}}{[(\beta\epsilon_w^\delta) + \epsilon_s^\delta]^2} -\dfrac{2\delta^2 \beta^{2\delta} \epsilon_w^{\delta} \epsilon_s^{2\delta-2}}{[(\beta\epsilon_w)^\delta + \epsilon_s^\delta]^3} < 0
\end{align}
After again applying Lemma 1 and noting that $\delta$ and $\beta$ are restricted to positive values, algebraic simplification shows that Equation \ref{eq:soc_s} holds if and only if $\beta > \left (\dfrac{\delta - 1}{3 - \delta} \right ) ^{\dfrac{1}{\delta}}$.

\subsection*{Proof of Lemma 2 (Equilibrium Strategy)}

Given Lemma 1 (Equal Effort), Equation \ref{fo_A} or \ref{fo_B} reduces to $\dfrac{\delta\beta^\delta\epsilon^{2\delta-1}}{(\beta^\delta + 1)^2\epsilon^{2\delta}} = 1$. Algebraic manipulation gives the equilibrium strategy, $s^*(\delta, \beta) = \epsilon = \delta\dfrac{\beta^\delta}{(1+ \beta^\delta)^2}$. But since the utility functions produce a discontinuous payoff space, I must also show that neither party as an incentive to defect from the proposed equilibrium to the boundary of zero effort. First, note by Lemma 1 that $u_w < u_s$ for all $\beta \in (0, 1)$ and $\delta >  1$. Therefore, if $w$ has no incentive to change from the proposed equilibrium, then neither does $s$. To see when $w$ has an incentive to change from the proposed equilibrium to zero effort, simply plug the proposed equilibrium effort into the utility function for $w$ and set the payoff function greater than or equal to zero. This shows that neither party has an incentive to switch to zero for $\beta \geq (\delta - 1)^{\dfrac{1}{\delta}}$. Any set of parameter values which satisfy this condition also satisfy the second order conditions. Therefore, the equilibrium exists if and only if neither party has an incentive to change from the proposed equilibrium strategy to zero effort.

\subsection*{Proof of Proposition 1 (Competitiveness)}

Given the equilibrium strategy of $s^*(\delta, \beta) = \epsilon = \delta\dfrac{\beta^\delta}{(\beta^\delta + 1)^2}$, I simply take the derivative of the equilibrium strategy function, $s^*$, with respect to the competitiveness parameter, $\beta$, to find how equilibrium mobilization efforts change as competitiveness changes.
\begin{equation}
\dfrac{\partial s^*}{\partial \beta} = \dfrac{\delta^2\beta^{\delta - 1}}{(\beta^\delta + 1)^2} - \dfrac{2\delta^2\beta^{2\delta - 1}}{(\beta^\delta + 1)^3}
\end{equation}
Algebraic manipulation yields $\dfrac{\partial s^*}{\partial \beta} = \dfrac{\delta^2\beta^{\delta - 1}(1 - \beta^\delta)}{(\beta^\delta + 1)^3}$. Since $\delta$ is defined to be greater than one and $\beta$ is defined to be between zero and one, this derivative is clearly positive. Thus, equilibrium effort is increasing in competitiveness.

\subsection{Proof of Proposition 2 (Disproportionality)}

Given the equilibrium strategy of $s^*(\delta, \beta) = \epsilon = \delta\dfrac{\beta^\delta}{(\beta^\delta + 1)^2}$, I simply take the derivative of the equilibrium strategy function, $s^*$, with respect to the disproportionality parameter, $\delta$, to find how equilibrium mobilization efforts change as disproportionality changes. This derivative is more complicated, but differentiation shows that $\dfrac{\partial s^*}{\partial \delta} = \dfrac{\beta^\delta + \delta\beta^\delta \text{ln}(\beta)}{(\beta^\delta + 1)^2} - \dfrac{2\beta^{2\delta}\text{ln}(\beta)}{(\beta^\delta + 1)^3}$. Algebraic manipulation shows that this derivative is positive if and only if $\delta\text{ln}{\beta} > \dfrac{\beta^\delta + 1}{\beta^\delta - 1}$. This derivative is positive for much of the parameter space for which the proposed equilibrium exists and only negative when both $\beta$ and $\delta$ are extremely small.

\subsection{Proof of Proposition 3 (Marginal Effect of Competitiveness)}

Given the equilibrium strategy of $s^*(\delta, \beta) = \epsilon = \delta\dfrac{\beta^\delta}{(\beta^\delta + 1)^2}$, I simply take the second derivative of the equilibrium strategy function, $s^*$, first differentiating with respect to the competitiveness parameter, $\beta$, and then with respect to the disproportionality parameter, $\delta$, to find how the marginal effect of competitiveness on equilibrium effort changes as disproportionality varies. This differentiation is complicated, but algebraic manipulation after setting it greater than zero yields the condition $\dfrac{1}{3} < \beta^\delta(1 + \beta^\delta)$. This inequality holds for almost all of the parameter space for which the pure strategy equilibrium exists.

\section{Extensions of the Formal Model}

\subsection{Expanding the Two-Party Model to the $n$-Party Case}

I briefly examine a less general model of party competition,but one that allows the value of the seat (or seats) in the contest to vary and allows more than two political parties. Making somehwat restrictive, but reasonable assumptions, I show that the key results from the two party model do not depend on the value of the prize (the number of seats competed over) or the number of parties in the competition. In this more complicated model, I assume that $n$ parties, where $P = \{1, 2,..., n\}$, compete over a seat or set of seats with value normalized to one. Each party places the same value on the seat (or set of seats). Again, I assume for simplicity that the prize is infinitely divisible. Similar to the simpler two-party model, I assume that the first $k$ parties are unable to mobilize voters as efficiently as the last $n - k$ parties. This gives two types of parties, $j \in \{w, s\}$, with the weak type, $w$, only able to mobilize a fraction of the voters that the strong type, $s$, is able to mobilize with the same amount of effort. Assuming risk neutrality gives the utility functions for each type.
\begin{align}
u_w(\epsilon_w, \epsilon_s) &= \dfrac{(\beta \epsilon_w)^\delta}{k(\beta \epsilon_w)^\delta + (n-k)\epsilon_s^\delta} - \epsilon_w \\
u_s(\epsilon_w, \epsilon_s) &= \dfrac{\epsilon_s^\delta}{k(\beta \epsilon_w)^\delta + (n-k)\epsilon_s^\delta} - \epsilon_s
\end{align}
\begin{definition}
\textbf{(Total Effort)} Define total effort, $\tau_\epsilon$, as the sum of the effort exerted by each party competing in the contest. Formally, $\tau_\epsilon = \sum_{i = 1}^n \epsilon_i$, for $i = \{1, 2,..., n\}$. 
\end{definition}
Notice that for the special case of two parties discussed above, the total effort exerted by the political parties is simply two times the equilibrium effort given the competitiveness and disproportionality of the contest. Formally, by Lemmas 1 and 2, $\tau_\epsilon(\delta, \beta) = 2\delta\dfrac{\beta^\delta}{(\beta^\delta + 1)^2}$ for the special case of a contest between only two political parties. I now use this definition to briefly expand beyond the simple model laid out above and explore the effects of multiple parties and varying district magnitude on the equilibrium effort.
\begin{lemma}
\textbf{(n-party Case)} The total effort, $\tau_\epsilon$, exerted in any contest does not depend on the number of parties in the contest. The number of parties in the contest only influences the amount of effort exerted by each individual party.
\end{lemma}
Proving Lemma 3 requires first finding a strategy profile for each type, $w$ and $s$ that satisfies the first and second order conditions of their respective utility functions. Second, it requires showing the the total effort, $\tau_\epsilon$, as defined in Definition 1 does not change as the number of parties in the contest varies. While the equilibrium effort may depend on the ratio of types (i.e. $\dfrac{k}{(n-k)}$), which is an aspect of competitiveness, it may not depend on the total number of parties in the contest (i.e. increasing the number of parties while leaving the ratio of type the same).

I begin by examining the first-order condition for the weak type's utility function. The utility function of the weak type is given by $u_w(\epsilon_w, \epsilon_s) = \dfrac{(\beta \epsilon_w)^\delta}{k(\beta \epsilon_w)^\delta + (n-k)\epsilon_s^\delta} - \epsilon_w$. Taking the derivative with respect to $\epsilon_w$, setting it equal to zero, and rearranging yields the first order condition for the weak type.
\begin{align}
\dfrac{(n-k)\delta\beta^\delta\epsilon_w^{\delta-1}\epsilon_s^\delta}{[k(\beta\epsilon_w)^\delta + (n-k)\epsilon_s^\delta]^2} = 1\label{fo_W}
\end{align}

I now examine the first-order conditions for the strong type's utility function. The utility function is given by $u_s(\epsilon_w, \epsilon_s) = \dfrac{\epsilon_s^\delta}{k(\beta \epsilon_w)^\delta + (n-k)\epsilon_s^\delta} - \epsilon_s$. Taking the derivative with respect to $\epsilon_s$, setting it equal to zero, and rearranging yields the first order condition for the strong type.
\begin{align}
\dfrac{k\delta\beta^\delta\epsilon_w^{\delta}\epsilon_s^{\delta-1}}{[k(\beta\epsilon_w)^\delta + (n-k)\epsilon_s^\delta]^2} = 1\label{fo_S}
\end{align}
Setting the left hand sides of Equations \ref{fo_W} and \ref{fo_S} equal to each other and algebraic manipulating the result shows that in equilibrium the strategies of the weak and strong types are not always equal, but have the relationship $\dfrac{k}{n-k}\epsilon_w = \epsilon_s$. 

Replacing $\epsilon_s$ in Equation \ref{fo_W} with $\dfrac{k}{n-k}\epsilon_w$ or replacing $\epsilon_w$ in Equation \ref{fo_S} with $\dfrac{n-k}{k}\epsilon_s$ and algebraically manipulating gives the identical equilibrium strategies for each type, $s^*_w$ and $s^*_s$.
\begin{align}
s^*_w(n, k, \delta, \beta) &= \dfrac{k\delta\beta^\delta(\dfrac{k}{n-k})^\delta}{[k\beta^\delta + (n-k)(\dfrac{k}{n-k})^\delta]^2} \\
s^*_s(n, k, \delta, \beta) &= \dfrac{k\delta\beta^\delta(\dfrac{k}{n-k})^{\delta+1}}{[k\beta^\delta + (n-k)(\dfrac{k}{n-k})^\delta]^2}
\end{align}
The above result gives the equilibrium effort for each individual party in the contest, depending on that party's type. All that remains is to show that these equilibrium strategies imply that the total effort, $\tau_\epsilon$ does not vary with the number of parties. By the definition of total effort given in Definition 1, the total effort in the n-party contest is given by $\tau_\epsilon = k \times \dfrac{k\delta\beta^\delta(\dfrac{k}{n-k})^\delta}{[k\beta^\delta + (n-k)(\dfrac{k}{n-k})^\delta]^2} + (n - k) \times \dfrac{k\delta\beta^\delta(\dfrac{k}{n-k})^{\delta+1}}{[k\beta^\delta + (n-k)(\dfrac{k}{n-k})^\delta]^2}$. Through algebraic manipulation, this reduces to $\tau_\epsilon = 2\delta\beta^\delta\dfrac{(\dfrac{k}{n-k})^{\delta - 1}}{[\beta^\delta + (\dfrac{k}{n-k})^{\delta-1}]^2}$. Clearly, the total effort does not depend on the number of parties, only on the ratio of weak and strong types.

\subsection{Expanding the Model to Allow for Prizes with Variable Value}

I now consider the consequences of changing the value of the prize over which the parties are competing. Again, I make restrictive but reasonable assumptions and show that the results from the simpler two-party model do not change. There are two methods of increasing the magnitudes of districts in electoral systems. First, policy-makers might lump districts together. This increases the district magnitude, but also makes the district larger, increasing the geographical size of the district and the population. This change makes mobilization more costly. I assume that doubling the size of the district doubles the cost of mobilization, that tripling the size of the district triples the cost of mobilization, and so on. Another method to increase district magnitude is to increase the size of the legislature. This seems to have little or no influence on the cost of mobilization, since the geographic size and population of the district remain the same. However, increasing the size of the legislature reduces the value of each seat. It is reasonable to assume that doubling the size of the legislature reduces the value of each seat by one-half, tripling the size of the legislature reduces the value of each seat by one-third, and so on. Either of these changes is captured by the assumption that the cost of effort is a function, $c$, of district magnitude, $m$, where $c(m, \epsilon)i) = m\epsilon_i$.
\begin{lemma}
\textbf{(Size of the Prize)} Assuming that the cost of effort is a function, $c$, of district magnitude, $m$, where $c(m, \epsilon_i) = m\epsilon_i$, the number of seats awarded in any contest does not influence the total effort exerted by the parties, $\tau_\epsilon$.
\end{lemma}
Changing the value of the seat or seats that parties compete over to $v$ rather than normalizing to one has no effect on the equilibrium efforts as long as reasonable adjustments are applied to the cost. As I argue in the main text, increasing the size of the district should increase the cost of mobilization. Assuming that a competition over a seat or seats $v$ times as valuable is $v$ times as costly, the equilibrium effort levels do not change. It is straightforward to see this in the general model or two-party. I show this for the strategy of the weak type, and the proof for the strong type is almost identical.  

The utility function of the weak type with variable values and associated changes in costs is given by $u_w(\epsilon_w, \epsilon_s) = v\dfrac{(\beta \epsilon_w)^\delta}{k(\beta \epsilon_w)^\delta + (n-k)\epsilon_s^\delta} - v\epsilon_w$. Taking the derivative with respect to $\epsilon_w$, setting it equal to zero, and rearranging yields the first order condition for the weak type.
\begin{equation}
v\dfrac{(n-k)\delta\beta^\delta \epsilon_w^{\delta-1}\epsilon_s^\delta}{[k(\beta\epsilon_w)^\delta + (n-k)\epsilon_s^\delta]^2} = v
\end{equation}
It is easy to see that the $v$'s on the left and right sides cancel, yielding a first-order condition that does not depend on $v$ and exactly the same first-order condition as the model without variable values.

\section{Robustness of Empirical Results}

In the first section of this appendix, I expand on the comments I make about the robustness of the empirical results in Footnote \ref{fn:robust} of the main text. I first show that the conclusions do not depend on whether I control for the number of political parties. Next, I show that the results are robust to complete pooling (single logit on the entire data set). Finally, I show that the results are robust to no pooling across countries (separate logits in each country). For computational ease, I (1) use listwise deletion for all the robustness checks, (2) assume that the random effects in the hierarchical models are independent, and (3) place Gamma(2, 0.01) priors on the standard deviation parameters of the hierarchical models (see \citealt{Chungetal2013}). Inferences do not depend on multiple imputation. Relaxing independence does not change any conclusions (and the estimated correlation in the main text is near zero). Placing priors on the standard deviation parameter gently pushes these parameters away from zero in a couple of instances. All explanatory variables in this section are scaled to have a minimum of zero and a maximum of one, with the exception of competitiveness, which has a theoretical range from zero to one but actually varies from 0.24 to 0.93. This scaling dramatically speeds the convergence of the MLE estimation.

\clearpage
\subsection{Controlling for Number of Political Parties Does Not Affect the Results}

One potential confound that I do not include in the main model is the number of political parties. Prior work shows that proportional electoral rules lead to more political parties, so it might be important to take this factor into account. Table \ref{tab:enep_partial} presents a series of hierarchical models that show the results do not change when I  include the effective number of political parties (at the district level) in the model or by including its interaction with electoral rules. The results are remarkably consistent across the different specifications.

Notice first that, as I expect, district competitiveness has a positive effect in SMDP systems (shown by the large, positive coefficient for competitiveness) and PR rules eliminate most of this positive effect (shown the large, negative coefficient for the product of competitiveness and PR rules). These findings hold regardless of the model specification. Next, notice that whether I include or exclude individual-level covariates in the hierarchical  models, the the model specification that I rely on in the main text (excluding the number of parties) minimizes the BIC, a criterion commonly used to choose model specifications. However, Table \ref{tab:F_partial} shows that the effect of the number of parties marginally statistically significant when individual-level controls are excluded from the model. But because the BIC points toward the model excluding the number of political parties, I present that model in the main text. However, the conclusions do not depend on this choice.

\input{tabs/enep_partial.tex}

 
 \begin{table}[h!]
 \begin{center}
 \begin{tabular}{c c c}
 Comparison & $F$-test & Direction \\
 Model 1 to Model 2 & $p = 0.09$ & + \\
 Model 1 to Model 3 & $p = 0.12$ & + \\
 Model 4 to Model 5 & $p = 1.00$ & 0 \\
 Model 4 to Model 6 & $p = 0.68$ & +/-- \\
 \end{tabular} \caption{This table compares models that include and exclude ENEP. The Comparison column reports which two models from Table \ref{tab:enep_partial} are being compared. The $F$-test column reports the $p$-value that that larger model (including ENEP) fits the data statistically significantly better than than the smaller model (excluding ENEP). The Direction column reports the direction of the estimated effect of ENEP.}\label{tab:F_partial}
 \end{center}
 \end{table}
 
\clearpage
\subsection{Estimating a Model with Complete Pooling Does Not Affect Results}

Table \ref{tab:enep_complete} presents estimates using a non-hierarchical logit model that completely pools the data across countries and districts. These results show that the conclusions in the main text and the conclusions about the limited impact of the number of parties on these results are robust to a complete pooling estimation strategy. Notice that across all model specifications, competitiveness has a large, positive effect in SMDP systems (shown by the coefficient for competitiveness) while PR rules consistently undermine this effect (shown by the large, negative coefficient for the product of competitiveness and PR rules).

By design, the empirical model I present in the main text partially pools information across districts. Because districts in Canada and Great Britain have fewer respondents and thus less information per district, there is more partial pooling in these districts. One might wonder whether this has an unintended effect on the results. However, the results from Table \ref{tab:enep_complete} using complete pooling closely mirror those from Table \ref{tab:enep_partial} using partial pooling. The substantive conclusions do not change.

However, Table \ref{tab:F_complete} shows that all specifications in Table \ref{tab:enep_complete} that include the number of political parties  fit the data statistically significantly better than models excluding the number of political parties. However, notice that signs of the estimated effects depend on whether I include individual-level covariates. Also again notice that the model specification that I rely on in the main text minimizes the BIC.

\input{tabs/enep_complete.tex}


 \begin{table}[h!]
 \begin{center}
 \begin{tabular}{c c c}
 Comparison & $F$-test & Direction \\
 Model 7 to Model 8 & $p = 0.01$ & + \\
 Model 7 to Model 9 & $p = 0.05$ & + \\
 Model 10 to Model 11 & $p = 0.01$ & -- \\
 Model 10 to Model 12 & $p = 0.04$ & -- \\
 \end{tabular} \caption{This table compares models that include and exclude ENEP. The Comparison column reports which two models from Table \ref{tab:enep_partial} are being compared. The $F$-test column reports the $p$-value that that larger model (including ENEP) fits the data statistically significantly better than than the smaller model (excluding ENEP). The Direction column reports the direction of the estimated effect of ENEP.}\label{tab:F_complete}
 \end{center}
 \end{table}

\clearpage
\subsection{Estimating a Model with No Pooling Across Countries Does Not Affect the Results}
 
The model I present in the main text assumes complete pooling across districts and elections. In this section, I also show that the results are robust to a no pooling assumption by estimating separate logistic regressions for each election in the data. This approach uses complete pooling across districts and no pooling across elections. The results are consistent with the main model (i.e. notice that competitiveness has a much larger effect in Canada and Great Britain). Table \ref{tab:CbyC_componly} presents results without individual-level controls, while Table \ref{tab:CbyC_full} includes them.


\input{tabs/CbyC_componly.tex}

\input{tabs/CbyC_full.tex}

I also replicate Figure \ref{fig:CbyC} from the main text using separate models in each country (i.e. the estimates presented in Table \ref{tab:CbyC_full}). Figure \ref{fig:CbyC_sep} shows these results.

\begin{figure}[h!]
\centering
\includegraphics[scale = .7]{figs/CbyC_sep.pdf}
\caption{This figure replicates Figure \ref{fig:CbyC} in the main text using separate logistic regression models in each election. That is, it exhibits no pooling across elections and complete pooling across districts. The results are very similar to Figure \ref{fig:CbyC} in the main text and the substantive conclusions are the same.}\label{fig:CbyC_sep}
\end{figure}

\clearpage

\section{Variance Priors}

The data I use to estimate the model parameters contain sufficient information to estimate all parameters reasonably well with the exception of the covariance matrix for the election-level effects. In this situation, the choice of prior can impact the results and one must choose a reasonable prior. A default, ``uninformative'' prior uniformly distributed from 0 to 100 yields results substantively similar to the results I discuss in the main text, but overestimates the variance parameters (\citealt{Gelman2006a}).

To alleviate concerns of overestimating variance parameters, I construct a weakly informative prior distribution for the diagonal entries of the covariance matrix for the election-level effects. The goal is to place a weakly-informative prior on the variance parameters (election-level variances, in this case) that contains much less prior information than is actually available, allowing the likelihood to drive the inferences. To do this, I use a half-Cauchy(5) distribution to model the standard deviations. Figure \ref{priors} shows that the half-Cauchy(5) is nearly uniform for reasonable values of the standard deviation, which for my purposes is roughly the interval [0,3]. Given the scale of the variables, it would be very strange if the true coefficients (conditional on electoral rules) across elections had a standard deviation of greater than three. One would probably expect a standard deviation of less than one. If the standard deviation were three, and the mean, for example, 2, then roughly half of the elections would have logit coefficients larger than four or less than zero. Given the scaling of the variables of interest, this is not a realistic scenario.

Another important property of the half-Cauchy(5) prior is that it goes to zero very slowly compared to the half-normal prior. So while it is roughly uniform from zero to three, it is quite flat out to ten. This property rules out very large standard deviations, such as 30, while allowing the likelihood to drive the estimates. Compare the half-Cauchy(5) prior to the uniform(0,100) and the half-normal(2). Notice that the uniform prior places an unrealistic, flat prior on the standard deviation, suggesting that large values near 50 are just as likely as small values near one. \cite{Gelman2006a} notes that the uniform(0,100) prior is problematic because it leads to an overestimation of the variance parameters, especially when the number of groups is small. The half-normal(2) prior, on the other hand, acts as a highly informative prior, ruling out large standard deviations and placing much higher prior weight on parameter values less than three. Although I use the half-Cauchy(5) prior in the main analysis, the results are substantively similar for both the uniform and the half-normal prior. 

\begin{figure}[h!]
\centering
\includegraphics[scale = .7]{figs/variance_priors.pdf}
\caption{This figure shows that the half-Cauchy(5) prior is a reasonable, weakly informative prior. It is roughly flat for reasonable standard deviations and somewhat flat for larger, but unrealistic values. \cite{Gelman2006a} points out that this prior structure will yield less upwardly biased estimates of standard deviation parameters. Compare the half-Cauchy(5) to the uniform(0,100), which places substantial prior weight on truly unrealistic standard deviations above ten, and the half-normal(2) prior, which places a much higher prior probability on parameter values less than three.}\label{priors}
\end{figure}

To illustrate that the weakly informative half-Cauchy prior is not driving the inferences, Figure \ref{postvar} shows the MCMC samples of the standard deviation parameter in the models using half-Cauchy(5), uniform(0,100), and half-normal(2) priors. In each case, plotting the prior is not useful because it appears as a barely visible flat line across the bottom of the histogram, suggesting the the likelihood is overwhelming the prior, as intended. However, notice that the posteriors based on the weakly informative half-Cauchy priors and the informative half-normal priors are more peaked and focused in a realistic region of the parameter space than the posteriors based on the uniform priors.

\begin{figure}[h!]
\centering
\includegraphics[scale = .7]{figs/variance_posteriors.pdf}
\caption{This figure shows that the half-Cauchy(5) prior does not overwhelm the likelihood. If the prior density were plotted, it would be a barely visible, flat line across the bottom of the plot. Further, all the posteriors are concentrated well below one, while both the half-Cauchy and uniform priors place a substantial prior probability above one.}\label{postvar}
\end{figure}

\clearpage
\section{Case Selection}

Table \ref{exclusion} lists each country in Module 2 of the CSES and explains if and why I exclude it from the sample.

\begin{table}[H]
{\scriptsize \begin{tabular}{ll}
\textbf{Country} & \textbf{Reason Excluded}\\
Albania & second-tier adjustments\\
Australia & concurrent upper house election \\
Belgium & concurrent upper house election\\
Brazil & concurrent presidential election, concurrent upper house election\\
Bulgaria & overlapping single and multimember districts \\
&(though not strictly a second-tier correction)\\
Canada & Included\\
Chile & concurrent presidential election, concurrent upper house election\\
Czech Republic & second-tier adjustment\\
Denmark & second-tier adjustment\\
Finland & Included\\
France & concurrent presidential election\\
Germany & second-tier adjustment\\
Great Britain & Included\\
Hong Kong & overlapping geographic and functional constituencies\\
Hungary & second-tier adjustment\\
Iceland & second-tier adjustment\\
Ireland & Droop Quota\\
Israel & single national district\\
Italy & concurrent upper house election\\
Japan & concurrent upper house election\\
Korea & second-tier correction\\
Kyrgyzstan & concurrent presidential election\\
Mexico & second-tier adjustment\\
Netherlands & single national district\\
New Zealand&second-tier adjustment\\
Norway&second-tier adjustment\\
Peru&concurrent presidential election\\
Philippines&concurrent presidential election\\
Poland&concurrent upper house election\\
Portugal&Included\\
Romania&concurrent presidential election, concurrent upper house election, \\
&  second-tier adjustment\\
Russia&concurrent presidential election\\
Slovenia&data unavailability\\
Spain&concurrent upper house elections\\
Sweden&second-tier adjustment\\
Switzerland&concurrent upper house elections\\
Taiwan&concurrent presidential election, second-tier adjustment\\
United States&concurrent presidential election, concurrent upper house election
\end{tabular}\caption{A table listing the countries in Module 2 of the CSES and explaining why each country was excluded.}\label{exclusion}}
\end{table}


\end{appendix}
\end{document}

